   \section*{Open Embedded Sound Server}

 \subsubsection*{\href{https://roseleblood.github.io/}{\tt Website}  $\vert$  \href{https://github.com/RoseLeBlood/openess/wiki}{\tt Wiki }  $\vert$  \href{https://github.com/RoseLeBlood/openess/blob/master/CONTRIBUTING.md}{\tt Contributing }  $\vert$  \href{https://roseleblood.github.io/openess/html/d3/dcc/md__r_e_a_d_m_e.html}{\tt Documentation }  $\vert$  \href{https://webchat.freenode.net/?channels=openess}{\tt Chat } }



 



{\bfseries Open\+E\+SS} is a network-\/capable sound server libary mainly for embedded systems and other operatins systems. Open\+E\+SS is free and open-\/source software, and is licensed under the terms of the G\+NU Lesser General Public License.

\href{https://www.codacy.com/app/RoseLeBlood/openess?utm_source=github.com&amp;utm_medium=referral&amp;utm_content=RoseLeBlood/openess&amp;utm_campaign=Badge_Grade}{\tt }

\subsection*{Getting Started}

\subsubsection*{Setting Up Development Environment}

\paragraph*{Platform\+IO}

Open\+E\+SS is made for use with \href{http://platformio.org/}{\tt platformio}, an advanced ecosystem for microcontroller development. To get started with coding esphomelib applications, you first have to \href{http://platformio.org/platformio-ide}{\tt install the atom-\/based platformio I\+DE} or for advanced users, \href{http://docs.platformio.org/en/latest/installation.html}{\tt install the command line version of platformio}.

Then create a new project for an \href{http://docs.platformio.org/en/latest/platforms/espressif32.html#boards}{\tt E\+S\+P32-\/based board} (for example, {\ttfamily nodemcu-\/32s}). Then open up the newly created {\ttfamily platformio.\+ini} file and insert


\begin{DoxyCode}
; ...
platform = espressif32
board = nodemcu-32s
framework = esp-idf
lib\_deps = openess
\end{DoxyCode}
 Finally, create a new source file in the {\ttfamily src/} folder (for example {\ttfamily main.\+c}) and start coding with openess.

\subsection*{Usage example}

{\itshape create the audio context} 
\begin{DoxyCode}
\textcolor{preprocessor}{#include "\hyperlink{ess__context_8h}{ess\_context.h}"}
\textcolor{preprocessor}{#include "\hyperlink{ess__format_8h}{ess\_format.h}"}

\textcolor{preprocessor}{#include <stdio.h>}
\textcolor{preprocessor}{#include <stdlib.h>}
\textcolor{preprocessor}{#include <unistd.h>}


\textcolor{keywordtype}{void} \hyperlink{esp32__create__context_8c_abce06be17fc37d675118a678a8100a36}{app\_main}() \{
  \hyperlink{structess__context}{ess\_context\_t} context;
  ess\_context\_error\_t error;

  error = \hyperlink{ess__context_8h_ab105a63f0da1117385f72c8d6530023c}{ess\_context\_create} (&context, 
      \hyperlink{config_8h_a02a4eb9ca75bf77fd60b4cb7bde37f2f}{ESS\_BACKEND\_NAME\_I2S\_ESP32}, \hyperlink{ess__format_8h_aa73ee1bade9564053c023e42c9ea1ec9ab4ad20e82e0980a848fb7fe80d0b3377}{ESS\_FORMAT\_STEREO\_44100\_16})
      ;
  \textcolor{keywordflow}{if}(error != ESS\_CONTEXT\_ERROR\_OK) printf(\textcolor{stringliteral}{"error in creating the context\(\backslash\)n"});

  \textcolor{keywordflow}{for}(;;) \{ usleep(100000); \}
\}
\end{DoxyCode}
 \+\_\+\+For more examples and usage, please refer to the \href{https://github.com/RoseLeBlood/openess/wiki}{\tt Wiki}

\subsection*{Current Features (version 0.\+2-\/1)}


\begin{DoxyItemize}
\item Powerful core that allows for easy to port
\item Automatic Wi\+Fi handling (reconnects, etc.)
\item Powerful socket abscrations layer (S\+AL)
\item Easy to use platform configuration
\item Semaphore, task and ringbuffer handling on various platform
\item generic backends\+: udp, uart and i2s
\end{DoxyItemize}

\subsection*{Progressed features (when ready than version 0.\+9)}


\begin{DoxyItemize}
\item running example server on esp32 and linux
\item audio mixing from multiple clients
\item M\+Q\+TT status upport and logging
\item m\+D\+NS
\item code style
\end{DoxyItemize}

\subsection*{Planned features}


\begin{DoxyItemize}
\item Improve documentation
\item {\bfseries Suggestions?} Feel free to create an issue and tag it with feature request.
\end{DoxyItemize}

\subsection*{Release History}


\begin{DoxyItemize}
\item 0.\+2
\begin{DoxyItemize}
\item rename headers
\item add platform abstraction layer
\end{DoxyItemize}
\item 0.\+0.\+1
\begin{DoxyItemize}
\item Work in progress 
\end{DoxyItemize}
\end{DoxyItemize}